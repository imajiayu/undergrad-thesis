\section{总结与讨论}\label{conclusion}

本文实现了一种基于体素的隐式表面表示的稠密建图方法。本文的模型不直接在体素网格中存储符号距离场,而是使用基于八叉树的隐式表示,同时将其中特征存储在哈希表中,最终通过神经网络转换为SDF值和语义标签。神经网络和特征可以从点云数据中端到端地进行学习。本文在合成和真实数据集上评估了该方法,并与当前最先进的建图系统相比较。本文的重建方法可以在大规模室外场景实现实时的几何与语义建模并达到具有优势的效果。针对无人驾驶仿真系统需求,本文方法可以为其提供高精度与分辨率的几何信息与语义信息,并在实时见图的条件下消耗较少的内存资源。

相较于Vox-Fusion,本文实现了在较为稀疏的激光雷达点云上大规模场景的三维重建,而非较稠密的RGB-D数据。本文证明了针对大规模场景,使用直接对采样点的符号距离场进行监督的方法在建图精细度与时间内存消耗上优于传统体渲染方法。最后,相较于现有的隐式建图方法,本文实现了较高精度的大规模语义建图。

本文在建图精细度和速度上仍有改进空间。如可以使用无锁结构的共享资源实现方法,使得多进程之间使用与更新资源更快;优化采样率与迭代次数,寻找时间与效果的平衡点。在几何精度上,可以将NeRF中的体渲染方法进行改进使得模型对表面的学习进一步加强,使用更先进的损失函数。在对抗网络遗忘上,可以使用正则项和关键帧重放机制相结合的方法,减少全局建图的次数,减少时间和资源消耗。进一步需要一些优化方法对于动态物体进行针对性的处理。目前基于神经网络的隐式建图方法存在训练复杂度较高,内存占用过大与泛化能力低等瓶颈,将所有信息使用隐式存储的方法会进一步加剧以上问题。显式与隐式相结合的存储与渲染方法作为一种在内存与建图精细度中的妥协方法,是隐式渲染未来优化的思路。
