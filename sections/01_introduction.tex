\section{引言}\label{introduction}

\subsection{问题背景}

同时定位与建图系统(Simultaneous Localization and Mapping, SLAM)在许多计算机视觉领域有重要的应用,如无人驾驶、室内机器人和虚拟现实等。无人驾驶系统需要高精度的地图,包括道路与车辆的几何信息等,帮助无人驾驶系统做出更准确的决策和规划路径。除精度外,一个能满足现实世界需求的,应用在无人驾驶领域的SLAM系统需要满足实时性,在大规模场下可扩展,最重要的是,对即便是没有观察到的地图区域,也有能力进行预测。

传统的SLAM方法一般以相机或激光雷达点云作为输入,可以满足在大规模场景下进行实时性应用的需要,并获得了较好的重建效果。然而,其无法对未观察到的区域做出合理的估计,因此无法得到这些位置大致的几何信息。与此同时,这些方式使用输入数据维护一个全局地图,并使用体素\cite{kahler}(voxel),代价体积\cite{learningDS}(cost volumes)或面元\cite{FusionDS}(surfels)等方法表示该地图的几何信息。在存储与使用这些地图时,高分辨率与保留三维细节信息往往意味着极高的内存代价。进一步,对这些地图进行局部更新是困难的,因为地图中的存储的元素数量过多且缺少关联信息。

随着NeRF\cite{nerf}的诞生,出现了一些基于学习的建图方法。由于NeRF本身可以用作新视角合成任务(Novel View Synthesis),这些方法可以在特定的数据集上进行训练并拥有一定程度的预测能力,并且在处理表面或边缘时展现出良好的效果。然而,原始NeRF本身针对的是小场景任务,例如对一个房间进行重建。这些方法存在训练时间过长的问题,并且在扩展到大规模场景时,这些方法的重建效果变得糟糕,并出现了神经网络中常见的灾难性遗忘(Catastrophic Forgetting)。与此同时,当前针对大规模场景的隐式重建方法往往忽略了场景中的语义信息。

针对目前存在的问题,以及为了能够适应无人驾驶应用的需要,本文提出了一种适用于大规模场景的可拓展的隐式建图方法,并实现了大规模语义建图。该方法基于体素,将显式存储与隐式存储相结合,以稀疏点云作为输入,将地图的几何与语义信息存储在体素网格中,使用多个解码器进行提取。该方法将几何信息编码为连续的符号距离场函数,使用基于表面的神经网络重建方法进行渲染。同时使用语义标签对网络与解码器进行监督训练,将语义信息嵌入至地图特征中。为进一步节省内存与时间消耗,该方法使用了基于八叉树的体素管理方法,可以实现地图的动态扩展和特征信息的快速查找。本文的系统使用多线程进行加速,并且使用了一个关键帧选择策略来克服遗忘问题。
\subsection{本文的工作}
本文的主要工作可以总结为以下4点:
\begin{itemize}
	\item 实现了基于八叉树体素的,可扩展的,分层的特征存储与采样方法,对全局地图进行动态的体素分配与管理,使用有限内存实现地图细节存储。
	\item 使用基于SDF的表面重建方法进行渲染,并实现了语义建图。本文使用带语义的点云数据对地图特征与解码器进行优化。
	\item 使用多线程对建图进行加速,不同线程之间共享数据,同时进行增量建图与全局建图。本文使用了一种关键帧选择策略来克服大规模场景下存在的遗忘问题。
	\item 在合成场景与真实世界场景的不同数据集中对本文方法进行了测试,并将结果提取为网格地图进行了定性和定量的评估。
\end{itemize}


\subsection{文章结构}
本文按照以下组织结构展开:第\ref{related work}章展示了传统方法与隐式场景表达相关工作的综述;第\ref{preliminaries}章给定了本文方法所使用到的预备知识;第\ref{algorithm}章具体描述了本文的方法;第\ref{implement}章描述了算法的具体实现;在第\ref{numerical experiments}章本文对该方法进行了评估与对比,并给出了一些实验细节;最终在第\ref{conclusion}章本文对该方法的限制与瓶颈做出了讨论。

\clearpage
\section{相关工作}\label{related work}

\subsection{传统视觉建图方法}
大部分视觉的SLAM系统以两种目标进行地图重建,一种构建稀疏地图用于定位,另一种构建稠密地图用来存储几何和语义信息。DTAM\cite{DTAM}率先使用了统一的稠密场景表示,以场景的深度图进行SLAM。稠密表示以稳定不变的方式实现跟踪和重新定位,作为一种与传感器无关的、统一的、完整的空间表示方法,具有开创性的意义。

稠密SLAM中的一些方法使用符号距离函数或面元(surfel)显式地表示表面,或使用占据概率间接存储几何信息。如果使用固定分辨率\cite{tradition1},那么这些表示在内存使用上的代价将非常昂贵。分层存储\cite{tradition2}的方法效率更高,但实现更为复杂,参数数量巨大,且细节只存储在小部分层级当中。

Voxblox\cite{voxblox}是第一种在动态增长的地图中从截断距离符号场TSDF (Truncated Signed Distance Function)增量构建欧式符号距离场ESDF (Euclidean Signed Distance Function)的方法。其实现了在大体素尺寸下最大化重建速度和表面精度,提供了定量的和实验上的关于ESDFs误差的分析,并在通过在无人机上使用这些地图以及在线再规划来验证整个系统。与之类似的VDB Fusion\cite{vdbfusion}的主要贡献是不需要对要映射的环境大小进行假设。

CodeSLAM\cite{CodeSLAM}与其相关工作如CodeMapping\cite{codemapping}等使用机器学习方法发现稠密结构的低维特征,从而实现高效的表示。但其使用深度图视图表示场景而不是完整的3D模型。
\subsection{神经隐式表达}
为了解决单个NeRF无法存储大规模场景的问题, Block-NeRF\cite{block}使用最直接的方式,即将大场景分为多个Block,单独训练NeRF网络,在推理时实现重组。该方法要求每个NeRF网络部分重叠,并且需要对切割部分进行优化训练。该方法是对NeRF进行可扩展化的初步尝试。

iMAP\cite{imap}是首个实现神经隐式表达的SLAM系统。给定一系列RGB-D图像, iMAP使用单个全连接层编码整个场景。该方法使用原始NeRF进行渲染,定位时固定场景通过反向传播进行位姿优化,渲染时对网络和位姿进行全局优化。由于单个网络容量有限, iMAP无法对更大的场景实现精确的建图,将所有信息存储在单个网络里也意味着无法实现扩展与局部更新。 

NICE-SLAM\cite{nice}与iMAP使用相同的系统结构与关键帧选择策略,在其基础上实现了地图的可扩展性。其使用了显式加隐式的方法,将地图特征存储在三个不同分辨率的网格中,并分别使用不同的解码器解码出占据概率。在渲染时,最低分辨率的特征用来预测墙壁,地面以及未观察到的区域,另外两个用来表示细节信息的网络相互进行优化。然而,在系统运行之前, NICE-SLAM需要对整个场景进行空间的分配,而在实际应用中是无法得知场景的确切边界的。进一步, NICE-SLAM需要对解码器进行预训练,降低了其可用性。

针对上述问题, SHINE-Mapping\cite{shine}和Vox-Fusion\cite{vox}均使用了可扩展的八叉树体素管理方法。SHINE-Mapping使用点云作为输入,使用一个全局解码器解码特征中的SDF值,并使用点云数据直接监督SDF进行训练,而非使用类似NeRF的方法模拟光线进行积分。其使用分层特征相叠加的方式,计算插值后,将不同层级八叉树的结果相加获得特征。为了克服遗忘问题,其使用和本次迭代有关参数数量的正则项进行优化。Vox-Fusion的系统结构和与NICE-SLAM相似,仍然使用RGB-D流作为输入。其解码器不需要与训练,使用深度对SDF值进行监督。其使用了基于体素的光线采样,相邻体素共享特征。相较于NICE-SLAM,其关键帧选择策略更简单有效。本论文参考了其中光线采样与关键帧选择的相关优化方法。

\subsection{语义建图}
SceneCode\cite{scenecode}扩展了CodeSLAM的方法,以此实现了语义建图。其能够通过优化多个帧之间的光度和语义标签的一致性来对改进网络预测的效率。然而,尽管使用深度图进行训练,但SceneCode仍然是基于视图的表示,并且缺乏对3D几何信息的真正感知。此外也有将语义信息嵌入其中的基于学习的方法\cite{semantic2},其使用线性分割渲染器在SRN上学习3D形状的外观和语义的联合隐式表示,并以半监督方式进行训练后。该网络可以从颜色或语义帧中合成带语义标签的新视图。

本文参考了Zhi等人提出的方法\cite{sem_nerf}。该方法在原始NeRF的网络结构中新增了关于语义的解码器,在计算出体密度后对特征再次使用两个解码器分别计算语义与颜色信息,增加语义相关的损失函数并使用原始方法进行训练。